%%%%%%%%%%%%%%%%%%%%%%%%%%%%%%%%%%%%%%%%%%%
%
% szablon pracy licencjackiej 
% korzystający ze stylu pracalicmgr.cls
% 2017.03.22 K. Turzynski
% 2017.03.01 P. Durka durka@fuw.edu.pl 
% na podstawie pliku J. Żygierewicz 2016
%
%%%%%%%%%%%%%%%%%%%%%%%%%%%%%%%%%%%%%%%%%%%



\documentclass{pracalicmgr}
\usepackage{polski}
\usepackage[utf8]{inputenc}

\author{$<$imię-i-nazwisko-Autora$>$}

\nralbumu{$<$numer-albumu$>$}

\title{$<$Tytuł pracy dyplomowej$>$}

\tytulang{$<$Tytuł pracy w tłumaczeniu na język angielski$>$}

\kierunek{$<$kierunek$>$}

\specjalnosc{$<$Specjalność-o-ile-dotyczy$>$}

\opiekun{$<$stopień naukowy, imię i nazwisko promotora$>$\\$<$Zakład-lub-Katedra$>$ \\ $<$Nazwa-Instytutu$>$}

%\dziedzina{13.200}
\dziedzina{13.2 Fizyka}

\date{$<$miesiąc-i-rok-złożenia-pracy$>$}

\keywords{$<$wykaz maksymalnie 10 słów swobodnie wybranych$>$}



\begin{document}


    \maketitle
    \let\cleardoublepage\clearpage
    
    \begin{abstract}
$<$Krótkie (maks. 800 znaków) streszczenie pracy, na przykład:

Lorem ipsum – tekst składający się z łacińskich i quasi-łacińskich wyrazów, mający korzenie w klasycznej łacinie, wzorowany na fragmencie traktatu Cycerona „O granicach dobra i zła” (De finibus bonorum et malorum) napisanego w 45 r. p.n.e. Tekst jest stosowany do demonstracji krojów pisma (czcionek, fontów), kompozycji kolumny itp. Po raz pierwszy został użyty przez nieznanego drukarza w XVI w.

Tekst w obcym języku pozwala skoncentrować uwagę na wizualnych aspektach tekstu, a nie jego znaczeniu.

Cytat z {\tt https://pl.wikipedia.org/wiki/Lorem\_ipsum}
$>$

     \end{abstract}

%    
%    \tableofcontents
%    
%    \chapter*{Cel pracy}
%    \addcontentsline{toc}{chapter}{Cel pracy}
%    \ldots 
%    \chapter{Wstęp}
%    Tutaj piszemy informacje wprowadzające w tematykę  pracy, potrzebne do zrozumienia treści.       
%        \section{Przykładowy podrozdział}
%        Przykładowa cytacja \cite{lamport86}.
%        
%    \chapter{Dane eksperymentalne}    
%    \chapter{Metodologia}
%    \chapter{Wyniki}
%    \chapter{Dyskusja}
%    \chapter{Podsumowanie}
%    \bibliographystyle{plain}
%    \bibliography{bibliografia}


\end{document}
